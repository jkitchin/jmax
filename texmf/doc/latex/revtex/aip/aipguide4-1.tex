%% ****** Start of file auguide.tex ****** %
%%
%%   This file is part of the AIP distribution of substyles for REVTeX 4.1
%%   For version 4.1r of REVTeX, August 2010
%%
%%   Copyright (c) 2009,2010 American Institute of Physics
%%
\listfiles
\documentclass[%
 reprint,%
%secnumarabic,%
 amssymb, amsmath,%
 aip,cha,%
%groupedaddress,%
%frontmatterverbose,
]{revtex4-1}

\usepackage{docs}%
\usepackage{bm}%
\usepackage[colorlinks=true,linkcolor=blue]{hyperref}%
%\nofiles
\expandafter\ifx\csname package@font\endcsname\relax\else
 \expandafter\expandafter
 \expandafter\usepackage
 \expandafter\expandafter
 \expandafter{\csname package@font\endcsname}%
\fi
\hyphenation{title}

\begin{document}

\title{Author's Guide to AIP Substyles for \revtex~4.1}%

\author{AIP Journal Program}%
\email{tex@aip.org}
\affiliation{American Institute of Physics\\Suite 1NO1, 2 Huntington Quadrangle\\Melville, New York 11747-4502, USA}%

\date{March 2010}%
\revised{August 2010}%

\maketitle

\tableofcontents

\section{Introduction}

This is the author's guide to the AIP substyles for \revtex~4.1, 
providing a useful formatting tool 
for \LaTeX\ users submitting papers to journals 
published by the American Institute of Physics.
This guide is intended as an adjunct to the documentation for \revtex\ itself 
(published by the American Physical Society), 
so information contained therein is not repeated here, 
except as it bears on the specific features of the AIP substyles.


\subsection{Prerequisite Documentation}

The following documentation should be considered your first source of information 
on how to prepare your document for use with this format; 
they are to be found within the APS \revtex~4.1 distribution. 
Updated versions of these are maintained at
the \revtex~4.1 homepage located at \url{http://authors.aps.org/revtex4/},
are also available at the Comprehensive \TeX\ Archive Network (CTAN, see \url{http://www.ctan.org/}), 
and form part of the \TeX Live distribution of \TeX.
\begin{itemize}
\item \textit{Author's Guide to \revtex~4.1}
\item \textit{\revtex~4.1 Command and Options Summary}
\item \textit{What's New in \revtex~4.1}
\end{itemize}
The present guide builds upon these documents, with which you should already be familiar.

The AIP substyles distribution for \revtex~4.1 includes 
a sample document (\file{aipsamp.tex}), 
a good starting point for 
the manuscript you are preparing for submission to an AIP journal.

By using \revtex's \textit{Author's Guide to \revtex~4.1}, you can develop your
document until it contains all of the content you desire.
This guide informs you on document class options, commands, and 
markup guidelines specific to AIP journals. 

\subsection{Software Requirements}

This guide assumes a working \revtex~4.1 installation including the AIP substyles. 
Please see the installation guide included with the distribution.\cite{Note1}

Please note that the AIP substyles work {\it only} with \revtex~4.1: 
the original \revtex~4.0 release does {\it not} make the AIP substyle available, nor is it compatible with them. 

For your computer to run \revtex~4.1 with the AIP substyles, the following are required:
\begin{itemize}
\item
a working installation of \LaTeX\,
\item
\revtex~4.1 and all packages it requires,
\item
the AIP substyles for \revtex~4.1, and
\item
any further \LaTeX\ packages used in your document. 
\end{itemize}

The easiest way to obtain all of the needed software is to install an up-to-date distribution of \TeX,
like \TeX Live, available on CTAN.

To obtain the most up-to-date version of this software, please see \url{http://www.aip.org/pubservs/compuscript.html}.


\subsection{Submitting to AIP Journals}

Authors preparing a manuscript for submission to
AIP journals should consult the Information for Contributors for the applicable journal,
available through links at \url{http://journals.aip.org/}. 
These requirements are not covered systematically in this author's guide; 
you are responsible for understanding the requirements of the particular journal to which
you will submit your article.

For further information about journal requirements, contact the Editorial
Office of the appropriate journal. (Follow links at \url{http://journals.aip.org/}.)

\subsection{Contact Information}\label{sec:resources}%
Any bugs, problems, or inconsistencies concerning the AIP journal substyles
should be reported to AIP support at \href{mailto:tex@aip.org}{tex@aip.org}. 
Reports should include information on the error and a 
\textit{small} sample document that manifests the problem, if possible. 
(Please don't send large files!) 

Feedback concerning \revtex~4.1 itself should be sent, as usual, 
to the American Physical Society at\\ \href{mailto:revtex@aps.org}{revtex@aps.org}. 

To determine if the problem you are experiencing belongs to \revtex\ or is specific to the
AIP substyles, simply remove \texttt{aip} from your document class options and rerun 
your document. If the problem goes away, you may assume that it is due to the AIP substyles;
if not, it belongs to \revtex.

\section{Sample \LaTeXe\ Document}
As the \revtex\ documentation makes clear, your document employs a \LaTeXe\ document class
(specifically \texttt{revtex4-1.cls}), so you should use 
the \LaTeXe\ commands and environments familiar to you with, say, the
standard article class \texttt{revtex4-1.cls}, and you will be able to 
employ many of the packages you are used to using with \LaTeXe.

Using \texttt{aipsamp.tex} as an example, 
your document will start with the usual \revtex\ \cmd\documentclass\ statement, but with
a particular document class option \texttt{aip} that specifies the AIP substyle:
\begin{verbatim}
\documentclass[aip]{revtex4-1}
\end{verbatim}
You will then invoke the \LaTeXe\-compatible packages your document requires, say:
\begin{verbatim}
\usepackage{graphicx}%
\usepackage{dcolumn}%
\usepackage{bm}%
\end{verbatim}
follow up with your document content:
\begin{verbatim}
\begin{document}
...
\end{verbatim}
and finish with a statement specifying your Bib\TeX\ database:
\begin{verbatim}
\bibliography{aipsamp}
\end{document}
\end{verbatim}

The books
in the bibliography of this guide provide extensive coverage of all topics
pertaining to preparing documents under \LaTeXe\; they are highly recommended. 

\section{\revtex\ Class Options Specific to AIP}

\subsection{Journal Substyle}
To access particular features of the AIP substyle, you will specify an additional document class option: the journal substyle, e.g.,
\begin{verbatim}
\documentclass[aip,jcp]{revtex4-1}
\end{verbatim}
in this case, \textit{J. Chem. Phys.}, the default. 
A complete list of AIP journals with the corresponding journal substyle appears in Table~\ref{tab:journals}.
\begin{table}
\caption{\label{tab:journals}AIP journal substyles}
\begin{ruledtabular}
\begin{tabular}{ll}
\textbf{Journal} & \textbf{class option} \\
\itshape Appl. Phys. Lett. &\texttt{apl}\\
\itshape Biomicrofluidics  &\texttt{bmf}\\
\itshape Chaos             &\texttt{cha}\\
\itshape J. Appl. Phys.    &\texttt{jap}\\
\itshape J. Chem. Phys.    &\texttt{jcp}\footnotemark[1]\\
\itshape J. Math. Phys.    &\texttt{jmp}\\
\itshape J. Renewable Sustainable Energy&\texttt{rse}\\
\itshape Phys. Fluids      &\texttt{pof}\\
\itshape Phys. Plasmas     &\texttt{pop}\\
\itshape Rev. Sci. Instrum.&\texttt{rsi}\\
\end{tabular}
\end{ruledtabular}
\footnotetext[1]{Default journal substyle.}
\end{table}

\subsection{Options for Citations and Bibliography}
The citation style for AIP journals is:
\begin{itemize}
\item 
numerical (default style), 
\item
author-year, and
\item
numerical author-year,
\end{itemize}
the latter two styles being only allowed for \textit{Chaos} or \textit{J. Math. Phys.}

The familiar numerical citations and numbered bibliography are the default for most journals: 
citations are superscript numbers, and the (numbered) bibliographic entries appear in the order cited. 

Author-year citations are only allowed for 
\textit{Chaos} or \textit{J. Math. Phys.}, with citations given in author-and-year format. 
Bibliographic entries are sorted by alphabetical order of first author's surname, then by year. 

Numerical author-year citations 
(only allowed for \textit{Chaos} or \textit{J. Math. Phys.}) 
are superscript numbers, just like numerical citations, 
but the bibliographic entries are sorted like the author-year entries and are numbered. 
This means that the first citation will not necessarily be~1.

To obtain the numerical style, simply accept the default, or supply a class option of \texttt{numerical}:
\begin{verbatim}
\documentclass[aip,numerical]{revtex4-1}
\end{verbatim}
For author-year citations for \textit{Chaos} or \textit{J. Math. Phys.}, 
you may specify the \texttt{author-year} option:
\begin{verbatim}
\documentclass[aip,author-year]{revtex4-1}
\end{verbatim}
Each of the above two options are part of standard \revtex.

To obtain numerical author-year citations 
for \textit{Chaos} or \textit{J. Math. Phys.}, 
give the author-numerical option:
\begin{verbatim}
\documentclass[aip,author-numerical]{revtex4-1}
\end{verbatim}
Note that the \texttt{author-numerical} option is not part of standard \revtex\, so use of it
outside of the AIP substyles may not have any effect. 

\subsection{Formatting Options}
There are two commonly used formats for an article you may write. 
One will comply with the manuscript submission formatting requirements of the editorial office of the journal you are submitting to.
The other will emulate the format of your article in the published journal itself. 

For journal submission, accept the default, or you may specify the \texttt{preprint} option:
\begin{verbatim}
\documentclass[aip,preprint]{revtex4-1}
\end{verbatim}
To emulate the formatting of the journal, specify the \texttt{reprint} option:
\begin{verbatim}
\documentclass[aip,reprint]{revtex4-1}
\end{verbatim}
Note that emulation is not by any means complete: the fonts used will differ, and therefore
the length of the article will not represent an accurate estimate. 
Other details may also differ. 

A summary of class options of interest to AIP authors appears in Table~\ref{tab:options}.
\begin{table}
\caption{\label{tab:options}Other class options}
\begin{ruledtabular}
\begin{tabular}{ll}
\textbf{Function} & \textbf{class option} \\
\multicolumn{2}{l}{\textit{Citation and References}}\\
superscript numbered&\texttt{numerical}\footnotemark[1]\textsuperscript{,}\footnotemark[2]\\
author-year&\texttt{author-year}\footnotemark[3]\\
numbered author-year&\texttt{author-numerical}\footnotemark[3]\\
%
\multicolumn{2}{l}{\textit{Format}}\\
journal submission&\texttt{preprint}\footnotemark[1]\\
journal emulation&\texttt{reprint}\\
\end{tabular}
\end{ruledtabular}
\footnotetext[1]{Default option.}%
\footnotetext[2]{Standard}%
\footnotetext[3]{Only allowed for \textit{Chaos} or \textit{J. Math. Phys.}}%
\end{table}

\section{Useful \LaTeXe\ Markup}
\LaTeXe\ markup is the preferred way to structure your file. 
In general, the use of low-level commands like \TeX\ primitives or Plain \TeX\ macros 
is less preferable. 
Please see the \revtex\ User's Guide,\cite{Note2} 
the \LaTeX\ manual,\cite{LaTeXman} 
and the \LaTeXe\ book\cite{Compan} 
for further details. 

\subsection{Title and Front Matter}\label{sec:front}

The \revtex\ User's Guide has complete information on using \revtex's special markup for your
article's title, author list, abstract, and other front matter elements. 
Note that class option \texttt{superscriptaddress} is the default for the AIP substyles, 
as required by all AIP journals. 

\subsection{Lead Paragraph}
One AIP journal, \textit{Chaos}, requires a paragraph of text to precede the first
\cmd\section\ of the article; 
this is known as a lead paragraph and is formatted boldface. 
To give your article a lead paragraph, 
include a quotation environment ahead of the first \cmd\section\ command:
\begin{verbatim}
\documentclass[aip]{revtex4-1}
\begin{document}
 \begin{quotation}
  Here is my lead paragraph!
 \end{quotation}
 \section{Introduction}
...
\end{verbatim}

The \texttt{quotation} environment functions normally after the first \cmd\section\ command in the document.

\section{Body}

For general information on commands used in the body of the document, see the \revtex\ User's Guide.
Herein are some features specific to the AIP author.

\subsection{Footnotes}

If you are using numbered citations (numerical or numbered author-year), 
footnotes are by default incorporated into the reference section 
along with your bibliographic entries. 
This automated feature is only effective if you use Bib\TeX\ to prepare your
bibliography. 

Author-year style bibliography does not lend itself to such a treatment, 
so by default footnotes appear in text as is usual. 
However, be advised that, if your article is accepted for publication,
footnotes may be incorporated into text during the production process.

\section{Citations and References}\label{sec:endnotes}

The preparation of your bibliography ``by hand'' is possible; 
however, if you do so, 
you will be entirely responsible 
for compliance with submission requirements for your bibliographic entries, 
for incorporating any text footnotes into the references, 
and for checking bibliographic entries. 
(In this connection, you may find useful the file \texttt{reftest.tex}, distributed with \revtex.)

There are numerous reasons to use Bib\TeX, not least because it automates the first and second of the above checks. 

\subsection{\label{sec:use-bib}Using Bib\protect\TeX}

Refer to the \revtex\ User's Guide, the \LaTeX\ manual, and the Bib\TeX\ manual
for full information about using Bib\TeX. 

When using Bib\TeX\, keep in mind that changing your bibliography style or citation style
(via the document class options described above) will require you to rerun Bib\TeX.
The standard litany (using \texttt{aipsamp.tex} as an example) for this is:
\begin{verbatim}
> latex aipsamp
> bibtex aipsamp
> latex aipsamp
> latex aipsamp
\end{verbatim}
Here, the first invocation of \texttt{latex} has the effect of rewriting the
\texttt{aipsamp.aux} file,
and the invocation of \texttt{bibtex} creates a new \texttt{aipsamp.bbl} file. 
The next two runs of \texttt{latex} are then required: 
the first to update the \texttt{aipsamp.aux} file reflecting the new values of your citations
and the second to employ those citations correctly. 
Be sure to check the end of the \texttt{aipsamp.log} file for any message advising you to 
rerun \texttt{latex}. 

\subsection{Multiple References per Citation}
In an article using numerical citations, 
it is not uncommon to encounter the need for a citation 
that refers to more than one article or other reference. 
To accommodate such a case, \revtex~4.1 implements markup similar to that of the 
\texttt{mcite} package for \LaTeXe. 

Let's say that two citation keys \texttt{able} and \texttt{baker} 
need to be combined into a single reference.
The syntax for the \cmd\cite\ command is:
\begin{verbatim}
word\cite{able,*baker} further text
\end{verbatim}
When you run Bib\TeX\, the resulting bibliography will contain the two entries, but run together
as a single numbered reference.
In the \cmd\cite\ command argument, any cite key that starts with the * character
signifies that its bibliographic entry is to be joined together with the one preceding it;
the \texttt{*} may join together any number of entries into a single reference.

\begin{thebibliography}{9}\label{sec:TeXbooks}%
\bibitem{Note1}
For help regarding the installation of this software and its use, please send email to \href{mailto:tex@aip.org}{tex@aip.org}.
%
\bibitem{Note2}
Available with the \revtex\ distribution, see \url{http://authors.aps.org/revtex4/}.
%
\bibitem[Lamport(1996)]{LaTeXman} 
L. Lamport, 
\emph{\LaTeX\, a Document Preparation System} 
(Addison-Wesley, Reading, MA, 1996).
%
\bibitem[Goossens(1994)]{Compan} 
M. Goosens, F. Mittelbach, and A. Samarin, 
\emph{The \LaTeX\ Companion} 
(Addison-Wesley, Reading, MA, 1994).
%
\bibitem[Knuth(1986)]{TeXbook} 
D. E. Knuth, 
\emph{The \TeX book} 
(Addison-Wesley, Reading, MA, 1986). 
%
\bibitem[Kopka(1995)]{Guide} 
H. Kopka and P. Daly, 
\emph{A Guide to \LaTeXe} 
(Addison-Wesley, Reading, MA, 1995).
%
\bibitem[Goossens(1997)]{CompanG} 
M. Goossens, S. Rahtz, and F. Mittelbach, 
\emph{The \LaTeX\ Graphics Companion} 
(Addison-Wesley, Reading, MA, 1997).
%
\bibitem[Rahtz(1999)]{CompanW} 
S. Rahtz, M. Goossens \emph{et al.},
\emph{The \LaTeX\ Web Companion} 
(Addison-Wesley, Reading, MA, 1999).
%
\end{thebibliography}

\end{document}

